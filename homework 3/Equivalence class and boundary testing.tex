\documentclass[12pt,letterpaper]{article}
\usepackage[utf8]{inputenc}
\usepackage{amsmath}
\usepackage{amsfonts}
\usepackage{amssymb}
\usepackage{array}
\usepackage[left=1.00in, right=1.00in, top=1.00in, bottom=1.00in]{geometry}
\author{Brett Worley}
\title{Homework 3: Equivalence class and Boundary testing}
\begin{document}

Brett Worley

CEG-3110-01

\centerline{Homework 3: Equivalence class and Boundary testing}

To implement equivalence classes and boundary testing, we will need to analyze the requirements
for the system and develop equivalence classes that will show the border values to test against. By going through the requirements for the system, we can identify the border conditions and come up with the following equivalence partitions shown in table 1.

% Possibly import the class partitions as figures from the powerpoint slides
\begin{table}[h!]
  \begin{center}
    \caption{Equivalence Partitions.}
    \label{tab:table1}
    
    % Length settings
    \begin{tabular}{||c|m{4cm}|m{5cm}||}
    \hline
    Less than 9 characters & At least 9 characters and not more than 24
    characters & More than 24 characters \\
    
    % Space settings
    \hline \hline
    \multicolumn{1}{||c|}{Contains no blank spaces} & \multicolumn{2}{c||}{Contains no blank spaces} \\
    
    % Lower case settings
    \hline \hline
    \multicolumn{1}{||c|}{Contains insufficient lower case} 
    & \multicolumn{2}{c||}{Contains sufficient lower case} \\
    
    % Upper case settings
    \hline \hline
    \multicolumn{1}{||c|}{Contains insufficient upper case} 
    & \multicolumn{2}{c||}{Contains sufficient upper case} \\
    
    % Number settings
    \hline \hline
    \multicolumn{1}{||c|}{Contains insufficient numbers} 
    & \multicolumn{2}{c||}{Contains sufficient numbers} \\
    
    % Special character settings
    \hline \hline
    \multicolumn{1}{||c|}{Contains insufficient special characters} 
    & \multicolumn{2}{c||}{Contains sufficient special characters} \\
    
    % Previous password settings
    \hline \hline
    \multicolumn{1}{||c|}{Contains a identical five-character substring}
    & \multicolumn{2}{c||}{Does not contain a indentical five-character subtring} \\
    \hline
    \end{tabular}
  \end{center}
\end{table}


% Table for the special characters allowed in the password program
\begin{table}[h!]
  \begin{center}
    \caption{Special Characters Allowed}
    \label{tab:specials}
    \begin{tabular}{|c|c|c|c|c|c|c|c|c|c|c|c|c|c|c|c|c|c|c|c|c|c|c|c|c|c|c|c|}
    \hline
    ! & @ & \# & \$ & \% & \& & * &  ) & ( & ] & [ & \} & \{ & $>$ & $<$ & ; 
    & : & . & , & / & $|$ &  \textbackslash & \~{} & ? & \_ & $-$ & $+$ & $=$ \\
    \hline
    \end{tabular}
  \end{center}
\end{table}

\newcommand{\specialcell}[2][c]{% Custom talbe lines, set for 3 rows in one table cell
  \begin{tabular}[#1]{@{}c@{}}#2\end{tabular}}

\begin{center}
  \begin{tabular}{||c|c||}
  \hline
  \multicolumn{2}{||c||}{Test Case 1} \\
  \hline
  Purpose & Testing a valid password \\
  \hline
  Input & DahatB2559\_@ \\
  \hline
  Previous Passwords & \specialcell{45tegrtdg\\23904vn239r8} \\% UolD2663!#
  \hline
  \end{tabular}
\end{center}

\end{document}