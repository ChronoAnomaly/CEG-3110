\documentclass[12pt,letterpaper]{article}
\usepackage[utf8]{inputenc}
\usepackage{amsmath}
\usepackage{amsfonts}
\usepackage{amssymb}
\usepackage{pgfplotstable} % Generates table from .csv
\usepackage{booktabs} % For \toprule, \midrule and \bottomrule
\usepackage[left=1.00in, right=1.00in, top=1.00in, bottom=1.00in]{geometry}
\author{Brett Worley}
\title{Orthogonal array}
\begin{document}

Brett Worley

CEG-3110-01

\centerline{ Homework 4: Orthogonal array}

In order to create a test plan from an orthogonal array, we need to look back at our equivalence classes
based off of the requirements. The orthogonal array method of creating test cases is based off of equivalence partitions
and the boundary cases that come from that. Using the orthogonal array method to generate test plans requires more thought
as to how to setup the program to actually generate an orthogonal array. While originally looking at how to layout
the sets of rules to create the orthogonal array, I had put each rule based on the equivalence partitions into
a 9 character password set. This setup was not usable, as the number of test cases created this way was too large
and a majority of the cases did not provide good testing or make much sense when looking at the requirements.


\begin{table}[h!]
  \begin{center}
    \caption{Equivalence Partitions.}
    \label{tab:Equiv}
    
    % Length settings
    \begin{tabular}{||c|m{4cm}|m{5cm}||}
    \hline
    Less than 9 characters & At least 9 characters and not more than 24
    characters & More than 24 characters \\
    
    % Space settings
    \hline \hline
    \multicolumn{1}{||c|}{Contains no blank spaces} & \multicolumn{2}{c||}{Contains no blank spaces} \\
    
    % Lower case settings
    \hline \hline
    \multicolumn{1}{||c|}{Contains insufficient lower case} 
    & \multicolumn{2}{c||}{Contains sufficient lower case} \\
    
    % Upper case settings
    \hline \hline
    \multicolumn{1}{||c|}{Contains insufficient upper case} 
    & \multicolumn{2}{c||}{Contains sufficient upper case} \\
    
    % Number settings
    \hline \hline
    \multicolumn{1}{||c|}{Contains insufficient numbers} 
    & \multicolumn{2}{c||}{Contains sufficient numbers} \\
    
    % Special character settings
    \hline \hline
    \multicolumn{1}{||c|}{Contains insufficient special characters} 
    & \multicolumn{2}{c||}{Contains sufficient special characters} \\
    
    % Previous password settings
    \hline \hline
    \multicolumn{1}{||c|}{Contains a identical five-character substring}
    & \multicolumn{2}{c||}{Does not contain a identical five-character substring} \\
    \hline
    \end{tabular}
  \end{center}
\end{table}

From our previous Equivalence classes, we can make the following border cases. These border conditions
are what we will use to setup rules for our orthogonal array generator in order to create usable test
cases to produce a good test plan.

\begin{enumerate}
\item Less than 9 characters
\item 9 characters
\item 24 characters
\item More than 24 characters
\item Contains insufficient lower case
\item Contains sufficient lower case
\item Contains insufficient upper case
\item Contains sufficient upper case
\item Contains insufficient numbers
\item Contains sufficient numbers
\item Contains insufficient special characters
\item Contains sufficient special characters
\item Contains a identical five-character substring (Similar)
\item Does not contain a identical five-character substring (Not similar)
\end{enumerate}

By using these border conditions, we are able to generate generic orthogonal array test cases.  The orthogonal arrays will be based on the setup
shown in table 2. Using a program to generate the orthogonal arrays, I created test plans that ran on different numbers of
strength for pairwise testing used to create orthogonal arrays. I tested n1, n2, n3, ,6, and n7 and made separate test plans for each
just to see how many test cases each plan contained. The n1 test plan consisted of 4 test cases. The n2 plan consisted of 11 test cases.
The n3 plan consisted of 27 cases. The n6 plan consisted of 184 cases. The n7 plan consisted of 256 cases. While this method produces
test cases that may not be thought of, it can generate some odd test plans. Such as in the n1, n2 test plans, which does not
contain a test case that actually passes the program based on the requirements. The n6 n7 generated plans produces at least one
test case that passes the program. In the following table I will list 8 of the possible test cases that resulted from the n3 orthogonal array.
This layout for the rule sets that creates an orthogonal array seems to produce decent test plans, although some of the test cases
do show failed passwords that will not pass the requirements but they are also similar to previous passwords.

\begin{table}[h!]
  \begin{center}
    \caption{Order of Orthogonal array results.}
    \label{tab:Array setup}
    
    % Length settings
    \begin{tabular}{||c|c||}
    \hline
	Number of characters & \# of characters \\
	\hline
	Contains a space & True/False \\
	\hline
	Sufficient Lower case & True/False \\
	\hline
	Sufficient Upper case& True/False \\
	\hline
	Sufficient numbers & True/False \\
	\hline
	Sufficient Special characters & True/False \\
	Similar & True/False \\
    \hline
    \end{tabular}
  \end{center}
\end{table}

Based on the previous table, I will list some examples from the n3 test plan that was generated by our orthogonal
array program. 

\begin{center}
    \begin{tabular}{||c|c||}
    \hline
	Number of characters & $<$9 \\
	\hline
	Contains a space & True \\
	\hline
	Sufficient Lower case & False \\
	\hline
	Sufficient Upper case& True \\
	\hline
	Sufficient numbers & True \\
	\hline
	Sufficient Special characters & False \\
	Similar & False \\
    \hline
    \end{tabular}
\end{center}
\begin{center}
    \begin{tabular}{||c|c||}
    \hline
	Number of characters & 9 \\
	\hline
	Contains a space & False \\
	\hline
	Sufficient Lower case & False \\
	\hline
	Sufficient Upper case& False \\
	\hline
	Sufficient numbers & False \\
	\hline
	Sufficient Special characters & True \\
	Similar & True \\
    \hline
    \end{tabular}    
\end{center}
\begin{center}
    \begin{tabular}{||c|c||}
    \hline
	Number of characters & 24 \\
	\hline
	Contains a space & False \\
	\hline
	Sufficient Lower case & True \\
	\hline
	Sufficient Upper case& True \\
	\hline
	Sufficient numbers & False \\
	\hline
	Sufficient Special characters & False \\
	Similar & True \\
    \hline
    \end{tabular}
\end{center}
\begin{center}
    \begin{tabular}{||c|c||}
    \hline
	Number of characters & $>$24 \\
	\hline
	Contains a space & True \\
	\hline
	Sufficient Lower case & False \\
	\hline
	Sufficient Upper case& False \\
	\hline
	Sufficient numbers & True \\
	\hline
	Sufficient Special characters & True \\
	Similar & True \\
    \hline
    \end{tabular}
\end{center}
\begin{center}
    \begin{tabular}{||c|c||}
    \hline
	Number of characters & 9 \\
	\hline
	Contains a space & True \\
	\hline
	Sufficient Lower case & True \\
	\hline
	Sufficient Upper case& False \\
	\hline
	Sufficient numbers & False \\
	\hline
	Sufficient Special characters & True \\
	Similar & False \\
    \hline
    \end{tabular}
\end{center}
\begin{center}    
    \begin{tabular}{||c|c||}
    \hline
	Number of characters & $<$9 \\
	\hline
	Contains a space & False \\
	\hline
	Sufficient Lower case & False \\
	\hline
	Sufficient Upper case& True \\
	\hline
	Sufficient numbers & True \\
	\hline
	Sufficient Special characters & True \\
	Similar & False \\
    \hline
    \end{tabular}
\end{center}
\begin{center}    
    \begin{tabular}{||c|c||}
    \hline
	Number of characters & 24 \\
	\hline
	Contains a space & False \\
	\hline
	Sufficient Lower case & True \\
	\hline
	Sufficient Upper case& True \\
	\hline
	Sufficient numbers & True \\
	\hline
	Sufficient Special characters & True \\
	Similar & False \\
    \hline
    \end{tabular}
\end{center}
\begin{center}    
    \begin{tabular}{||c|c||}
    \hline
	Number of characters & 24 \\
	\hline
	Contains a space & True \\
	\hline
	Sufficient Lower case & False \\
	\hline
	Sufficient Upper case& False \\
	\hline
	Sufficient numbers & False \\
	\hline
	Sufficient Special characters & False \\
	Similar & True \\
    \hline
    \end{tabular}
\end{center}



\end{document}