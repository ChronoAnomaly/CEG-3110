\documentclass[12pt,letterpaper]{article}
\usepackage[utf8]{inputenc}
\usepackage{amsmath}
\usepackage{amsfonts}
\usepackage{amssymb}
\usepackage{booktabs} % For \toprule, \midrule and \bottomrule
\usepackage[left=1.00in, right=1.00in, top=1.00in, bottom=1.00in]{geometry}
\author{Brett Worley}
\title{Decision table}



\begin{document}

Brett Worley

CEG-3110-01

\centerline{ Homework 5: Decision table}

In order to make a Decision table for testing, we need to get the basic logic of the system. We can take the system requirements and 
find the border cases for the system, this will lay out some of the conditions for
the system. By looking at the following cases, we can break down the cases to a smaller number
to be used in the Decision table.

\begin{enumerate}
\item Less than 9 characters
\item 9 characters
\item 24 characters
\item More than 24 characters
\item Contains insufficient lower case
\item Contains sufficient lower case
\item Contains insufficient upper case
\item Contains sufficient upper case
\item Contains insufficient numbers
\item Contains sufficient numbers
\item Contains insufficient special characters
\item Contains sufficient special characters
\item Contains a identical five-character substring (Similar)
\item Does not contain a identical five-character substring (Not similar)
\end{enumerate}

The broken down cases into simple logic for the Decision table would look like the following.
These cases would be True/False options that correspond to the logic of the system.

\begin{itemize}
\item 9-24 characters
\item Upper case
\item Lower case
\item Numbers
\item Special Characters
\item Space
\item Similar to previous password(s)
\end{itemize}

This list of the simplified logic could be used in the Decision table except that the similar to is vague based on the
requirements for the system. The requirements define similar to as: containing an identical five-character
substring (either forward or backward), independent of letter case.
Using this, we will break down the case into separate cases. While we could break down the case into
four separate cases (forward, backward, equals case, ignore case), the equals and
ignore cases would be redundant, as the ignore case would still catch situations that the equals
case is used for. The similar to we will broken into the two following separate cases in which the ignore case is implied.
\begin{itemize}
\item Similar five-character substring forwards
\item Similar five-character substring backwards
\end{itemize}

\begin{table}[h]
  \begin{center}
    \caption{Decision table}
    \label{tab:Main table}
    
    \begin{tabular}{c|c|c|c|c|c|c|c|c|c|c|}
    & 1 & 2 & 3 & 4 & 5 & 6 & 7 & 8 & 9 & 10 \\
    \hline
    9-24 chars & T & F & T & - & T & T & T & T & F & T \\
    \hline
    Upper case & T & T & T & - & T & T & F & F & T & T \\
    \hline
    Lower case & F & T & T & - & T & T & T & F & F & T \\
    \hline
    Numbers & T & T & T & - & T & T & T & T & F & T \\
    \hline
    Special characters & F & T & T & - & T & T & F & T & T & T \\
    \hline
    Space & F & F & F & T & F & F & T & F & F & F \\
    \hline 
    Similar to forwards & F & - & - & - & F & T & T & T & F & T \\
    \hline
    Similar to backwards & F & - & T & - & F & - & T & T & T & T \\
    \midrule
    Accept & & & & & x & & & & & \\
    \hline
    Reject & x & x & x & x & & x & x & x & x & x \\
    \end{tabular}
    
    Legend: T = True, F = False, - = Don't care, x = Expected output
  \end{center}
\end{table}

When making the Decision table I worked on mainly focusing on test cases that can trip up the
system possibly by using multiple types of similar to cases.  I also had worked in similar to cases
that are similar across different types of characters, this will allow any possible
flaws to be caught that occur in the logic of checking for similar cases in the
various passwords.

I had originally thought that, while the Decision table method could make making test cases easier based on the logic
of the system, that it was a little odd to do in the beginning.
After thinking about the first few test cases and putting them into the chart, I found that i was focusing more heavily on test cases
that could trip up the similar to logic.

\begin{table}[h]
  \caption{Test Plan}
  \label{tab:Test Plan}
\begin{centering}
  % test case 1
  \begin{tabular}{||c|c||}
  \hline
  \multicolumn{2}{||c||}{Test Case 1} \\
  \hline
  New Password & SOLID9019 \\
  \hline
  Current Password & catHOUsing24\_! \\
  \hline
  Previous Password & 9926CAdo\%\& \\
  \hline
  Expected Output & Rejected \\
  \hline
  \end{tabular}
  % test case 2
  \begin{tabular}{||c|c||}
  \hline
  \multicolumn{2}{||c||}{Test Case 2} \\
  \hline
  New Password & 2\$4(UpHi \\
  \hline
  Current Password & BONES@\#@house00 \\
  \hline
  Previous Password & 11Hou4singP\textbackslash= \\
  \hline
  Expected Output & Rejected \\
  \hline
  \end{tabular}
  % test case 3
  \begin{tabular}{||c|c||}
  \hline
  \multicolumn{2}{||c||}{Test Case 3} \\
  \hline
  New Password & 24RAThouses!\& \\
  \hline
  Current Password & OhtAR-+55Sesu \\
  \hline
  Previous Password & 440CAbin() \\
  \hline
  Expected Output & Rejected \\
  \hline
  \end{tabular}
  % test case 4
  \begin{tabular}{||c|c||}
  \hline
  \multicolumn{2}{||c||}{Test Case 4} \\
  \hline
  New Password & @@!90 horse \\
  \hline
  Current Password & NOTapassword01+= \\
  \hline
  Previous Password & oiadjoqwidjow \\
  \hline
  Expected Output & Rejected \\
  \hline
  \end{tabular}
  % test case 5
  \begin{tabular}{||c|c||}
  \hline
  \multicolumn{2}{||c||}{Test Case 5} \\
  \hline
  New Password & 2OH\_/5io \\
  \hline
  Current Password & 8888\#*daHI\\
  \hline
  Previous Password & CorectBat760=! \\
  \hline
  Expected Output & Accepted \\
  \hline
  \end{tabular}
  % test case 6
  \begin{tabular}{||c|c||}
  \hline
  \multicolumn{2}{||c||}{Test Case 6} \\
  \hline
  New Password &  ;:03Fires10001 \\
  \hline
  Current Password & Snake8,2Dl \\
  \hline
  Previous Password & 234+.;AntFires \\
  \hline
  Expected Output & Rejected \\
  \hline
  \end{tabular}
  % test case 7
  \begin{tabular}{||c|c||}
  \hline
  \multicolumn{2}{||c||}{Test Case 7} \\
  \hline
  New Password & 2240@racecar@ \\
  \hline
  Current Password & racecar 6480 \\
  \hline
  Previous Password & 772HArbor\_+ \\
  \hline
  Expected Output & Rejected \\
  \hline
  \end{tabular}
  % test case 8
  \begin{tabular}{||c|c||}
  \hline
  \multicolumn{2}{||c||}{Test Case 8} \\
  \hline
  New Password & !\%01010144444$-$$-$= \\
  \hline
  Current Password & $-$$-$=DarkHorse!\%444 \\
  \hline
  Previous Password & 992\^{}\_CallCenter \\
  \hline
  Expected Output & Rejected \\
  \hline
  \end{tabular}
  % test case 9
  \begin{tabular}{||c|c||}
  \hline
  \multicolumn{2}{||c||}{Test Case 9} \\
  \hline
  New Password & \$\$!@PROJECTHOUSES//=CELL\&\% \\
  \hline
  Current Password & !@pR344\$\%\&lLed=/ \\
  \hline
  Previous Password & 992\^{}\_CallCenter \\
  \hline
  Expected Output & Rejected \\
  \hline
  \end{tabular}
  % test case 10
  \begin{tabular}{||c|c||}
  \hline
  \multicolumn{2}{||c||}{Test Case 10} \\
  \hline
  New Password & 23AcrobatSstABorca!! \\
  \hline
  Current Password & orca28!\_Acrobatsstab \\
  \hline
  Previous Password & 992\^{}\_CallCenter \\
  \hline
  Expected Output & Rejected \\
  \hline
  \end{tabular}



\end{centering}


\end{table}
\end{document}
