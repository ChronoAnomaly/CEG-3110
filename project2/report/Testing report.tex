\documentclass[12pt,letterpaper]{article}
\usepackage[utf8]{inputenc}
\usepackage{amsmath}
\usepackage{amsfonts}
\usepackage{amssymb}
\usepackage{listings} % Source code formatting and highlighting
\usepackage{color}
\usepackage{pgfplotstable} % Generates table from .csv
\usepackage{booktabs} % For \toprule, \midrule and \bottomrule
\usepackage[left=1.00in, right=1.00in, top=1.00in, bottom=1.00in]{geometry}
\author{Brett Worley}
\title{Project 2: Password checker testing}
\date{version: \today}

\definecolor{dkgreen}{rgb}{0,0.6,0}
\definecolor{gray}{rgb}{0.5,0.5,0.5}
\definecolor{mauve}{rgb}{0.58,0,0.82}

\lstset{frame=,
  language=C,
  aboveskip=3mm,
  belowskip=3mm,
  showstringspaces=false,
  columns=flexible,
  basicstyle={\small\ttfamily},
  numbers=left,
  numberstyle=\color{black},
  keywordstyle=\color{blue},
  commentstyle=\color{dkgreen},
  stringstyle=\color{red},
  breaklines=true,
  breakatwhitespace=true,
  tabsize=4
}

\begin{document}

\maketitle
\tableofcontents



\section{Introduction}

The program to be developed is a password checking module. It will be used to check if a password
meets the requirements of the system. This document covers the various testing strategies
used to test the program. The finalized test plan was developed by using Equivalence class
and boundary testing, Orthogonal array testing, and Decision table testing. Along
with the code, a testing harness was created to provide automated and manual testing to check
the coverage of each test case. The purpose of using these various methods to test the program
is to ensure good coverage of the system from the finalized test plan.

\subsection{Requirements}

The requirements of the password checking program are as follows.

\begin{enumerate}

\item The new password shall be at least 9 characters long, and no longer than
24 characters.
\item New passwords cannot contain any blank spaces, and must contain only numberals, upper-case and lower-case letters, and
the specials characters listed in Requirement 4.
\item All new passwords must contain at least two upper case letters, at least two lower case
letters, and at least two numerals.
\item New passwords must contain at least two special characters from the following list: 
  !  @  \#  \$  \%  \&  *  )  (  ]  [  \}  \{  $>$  $<$  ; 
   :  .  ,  /  $|$   \textbackslash  \~{}  ?  \_  $-$  $+$  $=$ 

\item New passwords cannot be similar to any of the other two previous passwords, with ``similar to'' defined
as ``containing an identical five-character substring (either forward or backward), independent of letter
case (for example, a5\%Km and a5\%kM would be considered identical substrings).

\end{enumerate}

\subsection{Automated testing file format}

The testing harness program was developed to allow automated testing. It reads in a file given to the
program and will run through each test case, after each test case it will output whether it
passed or failed. The file to be read must match the following specified format for each test case.

\begin{center}
New password \\
Current password \\
Previous password \\
A or R (Case is ignored)
\end{center}

For this file format, the A or R line is for the expected output for each test case. The expect output
line also allows for a comment following the A or R, this is to help the tester keep track of
what is being tested against in this case. The file should contain the resulting final
test plan of 30 test cases, but any number of test cases will be allowed to be
read and processed.

\subsection{Manual testing}

The testing harness provides a means of manual testing. If the program is not passed a file
in its arguments, then it will proceed into manual entry mode. It will present the user
with a line on text stating how to exit this mode (the user must type exit), and then proceed
to wait on the password entries. The user will enter the password as such:
\begin{center}
New password \\
Current password \\
Previous password \\
\end{center}

This process will repeat until the user enters the previously mentioned string to exit this mode.
For each cycle of this, the program will process the given passwords and print out if the
new password is accepted or rejected, and then wait for more input.

\section{Equivalence class and boundary testing}

\subsection{Introduction}
To implement equivalence classes and boundary testing, we will need to analyze the requirements
for the system and develop equivalence classes that will show the border values to test against.
By going through the requirements for the system, we can identify the border conditions and
come up with the following equivalence partitions shown in table 1. The border conditions are important,
as they are the most likely place that the system will fail on and around.  Having these
cases also enables testers to trim down similar or equivalent test cases that essentially
test the same values, compared against the border conditions.

\subsection{Boundary conditions}

From the requirements of the program, we can form a set of boundary conditions. These
boundary conditions are the values that we will provides test for.

\begin{enumerate}
\item Contains less than 9 characters
\item Contains 9 characters
\item Contains 24 characters
\item Contains more than 24 characters
\item Contains 2 or more upper case
\item Contains 1 or less upper case
\item Contains 2 or more lower case
\item Contains 1 or less lower case
\item Contains 2 or more numbers
\item Contains 1 or less numbers
\item Contains 2 or more special characters
\item Contains 1 or less special characters
\item Contains no space
\item Contains a space
\item Contains a 5 character substring that is similar to a previous password
\begin{enumerate}
\item Similar five-character substring forwards
\item Similar five-character substring backwards
\end{enumerate}
\item Does not contain a 5 character substring that is similar to a previous password
\end{enumerate}


\subsection{Equivalence Partitions}

From the previous boundary conditions, we can form a set of equivalence classes. Each equivalence
class represents a set of data that is related.
% Possibly import the class partitions as figures from the powerpoint slides
\centering Equivalence Partitions.

\begin{center}

    
  % Length settings
  \begin{tabular}{||c|m{4cm}|m{5cm}||}
  \hline
  Less than 9 characters & At least 9 characters and not more than 24
  characters & More than 24 characters \\
  
  % Space settings
  \hline \hline
  \multicolumn{1}{||c|}{Contains no blank spaces} & \multicolumn{2}{c||}{Contains no blank spaces} \\
  
  % Lower case settings
  \hline \hline
  \multicolumn{1}{||c|}{Contains insufficient lower case} 
  & \multicolumn{2}{c||}{Contains sufficient lower case} \\
    
  % Upper case settings
  \hline \hline
  \multicolumn{1}{||c|}{Contains insufficient upper case} 
  & \multicolumn{2}{c||}{Contains sufficient upper case} \\
  
  % Number settings
  \hline \hline
  \multicolumn{1}{||c|}{Contains insufficient numbers} 
  & \multicolumn{2}{c||}{Contains sufficient numbers} \\
    
  % Special character settings
  \hline \hline
  \multicolumn{1}{||c|}{Contains insufficient special characters} 
  & \multicolumn{2}{c||}{Contains sufficient special characters} \\
  
  % Previous password settings
  \hline \hline
  \multicolumn{1}{||c|}{Contains a identical five-character substring}
  & \multicolumn{2}{c||}{Does not contain a indentical five-character subtring} \\
  \hline
  \end{tabular}
\end{center}


With these equivalence partitions, we are able to generate simple test cases
based on passing or failing the classes one at a time. The following test plan
is made up of 8 test cases that are designed to fail a certain requirement of
the system.
\newline


\subsection{Test plan}

\centering Equivalence class and Boundary Test Plan

\begin{center}
  \begin{tabular}{||c|c||}
  \hline
  \multicolumn{2}{||c||}{Test Case 1} \\
  \hline
  Purpose & Testing a valid password \\
  \hline
  New Password & DahatB2559\_@ \\
  \hline
  Current Password & ToT86635ss/\textless \\
  \hline
  Previous Password & AVery990\#\^{} \\
  \hline
  Expected Output & ACCEPTED \\
  \hline
  \end{tabular}
\end{center}
\vspace{1mm}
% Too Similar case
\begin{center}
  \begin{tabular}{||c|c||}
  \hline
  \multicolumn{2}{||c||}{Test Case 2} \\
  \hline
  Purpose & Testing against a similar password \\
  \hline
  New Password & SsaPmis628@@ \\
  \hline
  Current Password & \#558\#\&;DoGs \\
  \hline
  Previous Password & SimPass12!! \\
  \hline
  Expected Output & REJECTED: password too similar to a previous password \\
  \hline
  \end{tabular}
\end{center}
\vspace{1mm}
% Space fail case
\begin{center}
  \begin{tabular}{||c|c||}
  \hline
  \multicolumn{2}{||c||}{Test Case 3} \\
  \hline
  Purpose & Testing against a password containing a space\\
  \hline
  New Password & \&\^{}!aaCH91 chat \\
  \hline
  Current Password & \#558\#\&;DoGs \\
  \hline
  Previous Password & ToT86635ss/\textless \\
  \hline
  Expected Output & REJECTED: password contains a space \\
  \hline
  \end{tabular}
\end{center}
\vspace{1mm}
% Lower case test
\begin{center}
  \begin{tabular}{||c|c||}
  \hline
  \multicolumn{2}{||c||}{Test Case 4} \\
  \hline
  Purpose & Testing against a password with not enough lower case letters \\
  \hline
  New Password & \#\$678123HOUSE \\
  \hline
  Current Password & ToT86635ss/\textless \\
  \hline
  Previous Password & SimPass12!! \\
  \hline
  Expected Output & REJECTED: password does not have enough lower case letters \\
  \hline
  \end{tabular}
\end{center}
\vspace{1mm}
% Upper case test
\begin{center}
  \begin{tabular}{||c|c||}
  \hline
  \multicolumn{2}{||c||}{Test Case 5} \\
  \hline
  Purpose & Testing against password with not enough upper case letters \\
  \hline
  New Password & lowercasepasswords!;123 \\
  \hline
  Current Password & 445HAtter(( \\
  \hline
  Previous Password & \#558\#\&;DoGs \\
  \hline
  Expected Output & REJECTED: password does not have enough upper case letters \\
  \hline
  \end{tabular}
\end{center}
\vspace{1mm}
% Number test
\begin{center}
  \begin{tabular}{||c|c||}
  \hline
  \multicolumn{2}{||c||}{Test Case 6} \\
  \hline
  Purpose & Testing against not enough numbers \\
  \hline
  New Password & NumberBoycott\_? \\
  \hline
  Current Password & SimPass12!! \\
  \hline
  Previous Password & 445HAtter(( \\
  \hline
  Expected Output & REJECTED: password does not have enough numbers \\
  \hline
  \end{tabular}
\end{center}
\vspace{1mm}
% Short password test
\begin{center}
  \begin{tabular}{||c|c||}
  \hline
  \multicolumn{2}{||c||}{Test Case 7} \\
  \hline
  Purpose & Testing against a short password \\
  \hline
  New Password & Do12ah\_$=$ \\
  \hline
  Current Password & GUha891)) \\
  \hline
  Previous Password & \_CAThat22$=$ \\
  \hline
  Expected Output & REJECTED: password is too short \\
  \hline
  \end{tabular}
\end{center}
\vspace{1mm}
% Long password test
\begin{center}
  \begin{tabular}{||c|c||}
  \hline
  \multicolumn{2}{||c||}{Test Case 8} \\
  \hline
  Purpose & Testing against not enough numbers \\
  \hline
  New Password & WAYtooLong\_\_$+$$=$8835houseing40 \\
  \hline
  Current Password & \#558\#\&;DoGs \\
  \hline
  Previous Password & lowerCASEpasswords!;123 \\
  \hline
  Expected Output & REJECTED: password is too long \\
  \hline
  \end{tabular}
\end{center}



\section{Orthogonal array testing}

\subsection{Introduction}

In order to create a test plan from an orthogonal array, we need to look back at our equivalence classes
based off of the requirements. The orthogonal array method of creating test cases is based off of equivalence partitions
and the boundary cases that come from that. Using the orthogonal array method to generate test plans requires more thought
as to how to setup the program to actually generate an orthogonal array. While originally looking at how to layout
the sets of rules to create the orthogonal array, I had put each rule based on the equivalence partitions into
a 9 character password set. This setup was not usable, as the number of test cases created this way was too large
and a majority of the cases did not provide good testing or make much sense when looking at the requirements.


\subsection{Test method development}

\begin{center}

    
  % Length settings
  \begin{tabular}{||c|m{4cm}|m{5cm}||}
  \hline
  Less than 9 characters & At least 9 characters and not more than 24
  characters & More than 24 characters \\
  
  % Space settings
  \hline \hline
  \multicolumn{1}{||c|}{Contains no blank spaces} & \multicolumn{2}{c||}{Contains no blank spaces} \\
  
  % Lower case settings
  \hline \hline
  \multicolumn{1}{||c|}{Contains insufficient lower case} 
  & \multicolumn{2}{c||}{Contains sufficient lower case} \\
  
  % Upper case settings
  \hline \hline
  \multicolumn{1}{||c|}{Contains insufficient upper case} 
  & \multicolumn{2}{c||}{Contains sufficient upper case} \\
  
  % Number settings
  \hline \hline
  \multicolumn{1}{||c|}{Contains insufficient numbers} 
  & \multicolumn{2}{c||}{Contains sufficient numbers} \\
  
  % Special character settings
  \hline \hline
  \multicolumn{1}{||c|}{Contains insufficient special characters} 
  & \multicolumn{2}{c||}{Contains sufficient special characters} \\
   
  % Previous password settings
  \hline \hline
  \multicolumn{1}{||c|}{Contains a identical five-character substring}
  & \multicolumn{2}{c||}{Does not contain a identical five-character substring} \\
  \hline
  \end{tabular}
\end{center}
  
From our previous Equivalence classes, we can make the following border cases. These border conditions
are what we will use to setup rules for our orthogonal array generator in order to create usable test
cases to produce a good test plan.

\begin{enumerate}
\item Less than 9 characters
\item 9 characters
\item 24 characters
\item More than 24 characters
\item Contains insufficient lower case
\item Contains sufficient lower case
\item Contains insufficient upper case
\item Contains sufficient upper case
\item Contains insufficient numbers
\item Contains sufficient numbers
\item Contains insufficient special characters
\item Contains sufficient special characters
\item Contains a identical five-character substring (Similar)
\item Does not contain a identical five-character substring (Not similar)
\end{enumerate}

By using these border conditions, we are able to generate generic orthogonal array test cases.  The orthogonal arrays will be based on the setup
shown in table 2. Using a program to generate the orthogonal arrays, I created test plans that ran on different numbers of
strength for pairwise testing used to create orthogonal arrays. I tested n1, n2, n3, ,6, and n7 and made separate test plans for each
just to see how many test cases each plan contained. The n1 test plan consisted of 3 test cases. The n2 plan consisted of 11 test cases.
The n3 plan consisted of 27 cases. The n6 plan consisted of 195 cases. The n7 plan consisted of 288 cases. While this method produces
test cases that may not be thought of, it can generate some odd test plans. Such as in the n1, n2, n3 test plans, which does not
contain a test case that actually passes the program based on the requirements. The n6 n7 generated plans produces at least one
test case that passes the program. In the following table I will list the test plan that resulted from the n3 orthogonal array.
This layout for the rule sets that creates an orthogonal array seems to produce decent test plans, although some of the test cases
do show failed passwords that will not pass the requirements but they are also similar to previous passwords.
\newline

\centering Order of Orthogonal array results.
\begin{center}  
  % Length settings
  \begin{tabular}{||c|c||}
  \hline
  Number of characters & Low/Medium/High \\
  \hline
  Sufficient Lower case & True/False \\
  \hline
  Sufficient Upper case& True/False \\
  \hline
  Sufficient numbers & True/False \\
  \hline
  Sufficient Special characters & True/False \\
  \hline
  Similar & False/Forward/Backward \\
  \hline
  Contains a space & True/False \\
  \hline
  \end{tabular}
\end{center}


\subsection{Test plan}

\begin{tabular}{|r|c|c|c|c|c|c|c|}
\hline
\multicolumn{1}{|l|}{Test Case} & Num Chars & Upper & Lower & Numbers & Special & Space & Similar  \\ \hline
1                               & low       & False & False & False   & False   & True  & False    \\ \hline
2                               & medium    & True  & True  & True    & True    & False & Forward  \\ \hline
3                               & high      & True  & False & True    & False   & True  & Backward \\ \hline
4                               & medium    & False & False & False   & True    & False & Backward \\ \hline
5                               & low       & False & True  & False   & False   & True  & Forward  \\ \hline
6                               & high      & False & True  & True    & True    & False & False    \\ \hline
7                               & low       & True  & True  & True    & True    & True  & Backward \\ \hline
8                               & medium    & True  & False & False   & False   & False & False    \\ \hline
9                               & high      & True  & False & False   & True    & False & Forward  \\ \hline
10                              & low       & False & True  & True    & False   & False & Backward \\ \hline
11                              & medium    & False & False & True    & False   & True  & Forward  \\ \hline
12                              & low       & True  & True  & False   & True    & True  & False    \\ \hline
13                              & high      & True  & True  & False   & False   & False & Backward \\ \hline
14                              & high      & False & False & True    & True    & True  & Forward  \\ \hline
15                              & low       & True  & False & True    & True    & False & False    \\ \hline
16                              & medium    & True  & True  & True    & False   & True  & False    \\ \hline
17                              & high      & False & False & False   & False   & True  & False    \\ \hline
18                              & medium    & False & True  & False   & False   & True  & Backward \\ \hline
19                              & low       & True  & False & False   & False   & False & Forward  \\ \hline
20                              & high      & True  & True  & True    & False   & True  & Forward  \\ \hline
21                              & medium    & True  & True  & True    & True    & True  & Backward \\ \hline
22                              & low       & False & False & False   & True    & True  & Backward \\ \hline
23                              & medium    & False & False & False   & False   & False & Forward  \\ \hline
24                              & high      & False & True  & False   & True    & True  & Backward \\ \hline
25                              & medium    & False & False & True    & True    & False & False    \\ \hline
26                              & low       & False & False & True    & True    & False & Forward  \\ \hline
27                              & high      & True  & False & False   & False   & False & False    \\ \hline
\end{tabular}


\section{Decision table testing}

\subsection{Introduction}

In order to make a Decision table for testing, we need to get the basic logic of the system.
We can take the system requirements and find the border cases for the system, this will lay
out some of the conditions for the system. By looking at the following cases, we
can break down the cases to a smaller number to be used in the Decision table.

The decision table method of testing is more involved on thinking on the basic cases to create
test cases. It breaks down the test cases into a semi-truth table format and allows you
to think more carefully about the logic of what should happen in the system depending on what
flags (values in the system) are true or false.

\subsection{Test method development}

\begin{enumerate}
\item Less than 9 characters
\item 9 characters
\item 24 characters
\item More than 24 characters
\item Contains sufficient lower case
\item Contains sufficient upper case
\item Contains sufficient numbers
\item Contains sufficient special characters
\item Contains a identical five-character substring (Similar)
\item Does not contain a identical five-character substring (Not similar)
\end{enumerate}

This list of the simplified logic could be used in the Decision table except that the similar to is vague based on the
requirements for the system. The requirements define similar to as: containing an identical five-character
substring (either forward or backward), independent of letter case.
Using this, we will break down the case into separate cases. While we could break down the case into
four separate cases (forward, backward, equals case, ignore case), the equals and
ignore cases would be redundant, as the ignore case would still catch situations that the equals
case is used for. The similar to we will broken into the two following separate cases in which the ignore case is implied.

\begin{itemize}
\item Similar five-character substring forwards
\item Similar five-character substring backwards
\end{itemize}

The broken down cases into simple logic for the Decision table would look like the following.
These cases would be True/False options that correspond to the logic of the system.

\begin{itemize}
\item less than 9 characters
\item 9 - 24 characters
\item more than 24 characters
\item Upper case
\item Lower case
\item Numbers
\item Special Characters
\item Space
\item Similar five-character substring forwards
\item Similar five-character substring backwards
\end{itemize}



\subsection{Resulting Decision table}

% Decision table part

\begin{center}

    
  \begin{tabular}{c|c|c|c|c|c|c|c|c|c|c|c|}
  & 1 & 2 - 26 & 26 & 27 - 63 & 64 - 100 & 101 & 102 & 103 & 104 & 105 & 106 \\
  \hline
  less than 9 chars & T &  & F &  &  & F & F & F & F & F & F \\
  \hline
  9-24 chars & T & F & T &  &  & T & T & T & T & T & T \\
  \hline
  more than 24 chars & T &  & F & & & F & F & F & F & F & F \\
  \hline
  Upper case & F & - & F &  &  & T & - & T & F & T & T \\
  \hline
  Lower case & F & - & F &  &  & T & - & T & T & F & T \\
  \hline
  Numbers & F & - & F &  &  & T & - & F & T & T & F \\
  \hline
  Special characters & F & - & T &  &  & T & - & T & T & F & F \\
  \hline
  Space & F & - & T &  &  & F & - & F & F & F & F \\
  \hline 
  Similar to, forwards & T & F & F & T & F & F & T & F & F & F & F \\
  \hline
  Similar to, backwards & F & F & T & F & T & F & T & F & F & F & F \\
  \midrule
  Accept & & & & & & X & & & & & \\
  \hline
  Reject & & X & X & X & X & & X & X & X & X & X \\
  \hline
  Impossible & X & & & & & & & & & & \\
  \end{tabular}
    
  Legend: T = True, F = False, - = Don't care, x = Expected output
\end{center}


\subsection{Test plan}

\centering Decision table Test Plan
% Decision table test cases

\begin{center}

  
  \begin{tabular}{||c|c||}
  \hline
  \multicolumn{2}{||c||}{Test Case 1} \\
  \hline
  New Password & SOLID9019 \\
  \hline
  Current Password & catHOUsing24\_! \\
  \hline
  Previous Password & 9926CAdo\%\& \\
  \hline
  Expected Output & Rejected \\
  \hline
  \end{tabular}
  \vspace{1mm}
  \begin{tabular}{||c|c||}
  \hline
  \multicolumn{2}{||c||}{Test Case 2} \\
  \hline
  New Password & 34ha\$=AT \\
  \hline
  Current Password & COrrect440@\# \\
  \hline
  Previous Password & HousIng99!! \\
  \hline
  Expected Output & Rejected \\
  \hline
  \end{tabular}
  \begin{tabular}{||c|c||}
  \hline
  \multicolumn{2}{||c||}{Test Case 3} \\
  \hline
  New Password & !!\_=\#\# \$ ++ \\
  \hline
  Current Password & 20CatH!a\^{}ts \\
  \hline
  Previous Password & HIca=\_!!++113 \\
  \hline
  Expected Output & Rejected \\
  \hline
  \end{tabular}
  \vspace{1mm}
  \begin{tabular}{||c|c||}
  \hline
  \multicolumn{2}{||c||}{Test Case 4} \\
  \hline
  New Password &  256TieE44472\^{}! \\
  \hline
  Current Password & 992\^{}\_CallCenter \\
  \hline
  Previous Password & 00\^{}!IE44472hats \\
  \hline
  Expected Output & Rejected \\
  \hline
  \end{tabular}
  \begin{tabular}{||c|c||}
  \hline
  \multicolumn{2}{||c||}{Test Case 5} \\
  \hline
  New Password &  ;:03Fires10001 \\
  \hline
  Current Password & Snake8,2Dl \\
  \hline
  Previous Password & 234+.;SErifant \\
  \hline
  Expected Output & Rejected \\
  \hline
  \end{tabular}
  \vspace{1mm}
  \begin{tabular}{||c|c||}
  \hline
  \multicolumn{2}{||c||}{Test Case 6} \\
  \hline
  New Password &  ValidPass\#!110 \\
  \hline
  Current Password & 992\^{}\_CallCenter \\
  \hline
  Previous Password & 20CatH!a\^{}ts \\
  \hline
  Expected Output & Accept \\
  \hline
  \end{tabular}
  \begin{tabular}{||c|c||}
  \hline
  \multicolumn{2}{||c||}{Test Case 7} \\
  \hline
  New Password & 23AcrobatSstABorca!! \\
  \hline
  Current Password & orca28!\_Acrobatsstab \\
  \hline
  Previous Password & 992\^{}\_CallCenter \\
  \hline
  Expected Output & Rejected \\
  \hline
  \end{tabular}
  \vspace{1mm}
  \begin{tabular}{||c|c||}
  \hline
  \multicolumn{2}{||c||}{Test Case 8} \\
  \hline
  New Password & Housing\^{}.House \\
  \hline
  Current Password & orca28!\_Acrobatsstab \\
  \hline
  Previous Password & ValidPass\#!110 \\
  \hline
  Expected Output & Rejected \\
  \hline
  \end{tabular}
  \begin{tabular}{||c|c||}
  \hline
  \multicolumn{2}{||c||}{Test Case 9} \\
  \hline
  New Password & no!\#uppers24 \\
  \hline
  Current Password & \#Dog55Cat! \\
  \hline
  Previous Password & H234\^{}.House \\
  \hline
  Expected Output & Rejected \\
  \hline
  \end{tabular}
  \vspace{1mm}
  \begin{tabular}{||c|c||}
  \hline
  \multicolumn{2}{||c||}{Test Case 10} \\
  \hline
  New Password & WHY3456LOWER \\
  \hline
  Current Password & NO10!\#uppers24 \\
  \hline
  Previous Password & \#Dog55Cat! \\
  \hline
  Expected Output & Rejected \\
  \hline
  \end{tabular}
  \begin{tabular}{||c|c||}
  \hline
  \multicolumn{2}{||c||}{Test Case 11} \\
  \hline
  New Password & WORKpassword \\
  \hline
  Current Password & 5Dark4Hors\#e= \\
  \hline
  Previous Password & HoRs456-+ \\
  \hline
  Expected Output & Rejected \\
  \hline
  \end{tabular}

\end{center}



\section{Combined Test Plan}

\subsection{Results}

From the various testing methods that were used for testing this program,
we can combine the resulting test cases into a single test plan. This final
test plan will provide comprehensive coverage of the program. The resulting test plan
size contains a total of 30 test cases. As the more complex part of the system is checking
if a password is similar to a previous password, this section will be tested more heavily
from the test cases.
\newline\newline
% Finalized test plan table area
\centering Final Test Plan

\begin{center}
  \begin{tabular}{|c|c|}
  \hline
  \multicolumn{2}{|c|}{Test Case 1} \\
  \hline
  Purpose & \begin{tabular}[c]{@{}l@{}}The purpose of this test case is\\ to test the ability of the\\ program to reject a password\\ that is similar\end{tabular}  \\
  \hline
  New Password & SsaPmis628@@ \\
  \hline
  Current Password & \#558\#\&;DoGs \\
  \hline
  Previous Password & SimPass12!! \\
  \hline
  Expected Output & Rejected \\
  \hline
  Output & \\
  \hline
  Results & \multicolumn{1}{c|}{$\rule{0.5cm}{0.15mm}$Pass $\rule{0.5cm}{0.15mm}$Fail} \\ 
  \hline
  \end{tabular}
  \vspace{5mm}
  \begin{tabular}{|c|c|}
  \hline
  \multicolumn{2}{|c|}{Test Case 2} \\
  \hline
  Purpose & \begin{tabular}[c]{@{}l@{}}The purpose of this test case is\\ to test the ability of the\\ program to reject a password\\ that has a space\end{tabular}  \\
  \hline
  New Password & \&\^{}!aaCH91 chat \\
  \hline
  Current Password & \#558\#\&;DoGs \\
  \hline
  Previous Password & ToT86635ss/\textless \\
  \hline
  Expected Output & Rejected \\
  \hline
  Output & \\
  \hline
  Results & \multicolumn{1}{c|}{$\rule{0.5cm}{0.15mm}$Pass $\rule{0.5cm}{0.15mm}$Fail} \\ 
  \hline
  \end{tabular}
  \vspace{5mm}
  \begin{tabular}{|c|c|}
  \hline
  \multicolumn{2}{|c|}{Test Case 3} \\
  \hline
  Purpose & \begin{tabular}[c]{@{}l@{}}The purpose of this test case is\\ to test the ability of the\\ program to reject a password\\ that has not enough lower case\end{tabular}  \\
  \hline
  New Password & \#\$678123HOUSE \\
  \hline
  Current Password & ToT86635ss/\textless \\
  \hline
  Previous Password & SimPass12!! \\
  \hline
  Expected Output & Rejected \\
  \hline
  Output & \\
  \hline
  Results & \multicolumn{1}{c|}{$\rule{0.5cm}{0.15mm}$Pass $\rule{0.5cm}{0.15mm}$Fail} \\ 
  \hline
  \end{tabular}
  \vspace{5mm}
  \begin{tabular}{|c|c|}
  \hline
  \multicolumn{2}{|c|}{Test Case 4} \\
  \hline
  Purpose & \begin{tabular}[c]{@{}l@{}}The purpose of this test case is\\ to test the ability of the\\ program to reject a password\\ that has not enough upper case\end{tabular}  \\
  \hline
  New Password & lowercasepasswords!;123 \\
  \hline
  Current Password & 445HAtter(( \\
  \hline
  Previous Password & \#558\#\&;DoGs \\
  \hline
  Expected Output & Rejected \\
  \hline
  Output & \\
  \hline
  Results & \multicolumn{1}{c|}{$\rule{0.5cm}{0.15mm}$Pass $\rule{0.5cm}{0.15mm}$Fail} \\ 
  \hline
  \end{tabular}
  \vspace{5mm}
  \begin{tabular}{|c|c|}
  \hline
  \multicolumn{2}{|c|}{Test Case 5} \\
  \hline
  Purpose & \begin{tabular}[c]{@{}l@{}}The purpose of this test case is\\ to test the ability of the\\ program to reject a password\\ that is too short\end{tabular}  \\
  \hline
  New Password & Do12ah\_= \\
  \hline
  Current Password & GUha891)) \\
  \hline
  Previous Password & \_CAThat22= \\
  \hline
  Expected Output & Rejected \\
  \hline
  Output & \\
  \hline
  Results & \multicolumn{1}{c|}{$\rule{0.5cm}{0.15mm}$Pass $\rule{0.5cm}{0.15mm}$Fail} \\ 
  \hline
  \end{tabular}
  \vspace{5mm}
  \begin{tabular}{|c|c|}
  \hline
  \multicolumn{2}{|c|}{Test Case 6} \\
  \hline
  Purpose & \begin{tabular}[c]{@{}l@{}}The purpose of this test case is\\ to test the ability of the\\ program to reject a password\\ that is too long\end{tabular}  \\
  \hline
  New Password & WAYtooLong\_+=8835houseing40 \\
  \hline
  Current Password & \#558\#\&;DoGs \\
  \hline
  Previous Password & lowerCASEpasswords!;123 \\
  \hline
  Expected Output & Rejected \\
  \hline
  Output & \\
  \hline
  Results & \multicolumn{1}{c|}{$\rule{0.5cm}{0.15mm}$Pass $\rule{0.5cm}{0.15mm}$Fail} \\ 
  \hline
  \end{tabular}
  \vspace{5mm}
  \begin{tabular}{|c|c|}
  \hline
  \multicolumn{2}{|c|}{Test Case 7} \\
  \hline
  Purpose & \begin{tabular}[c]{@{}l@{}}The purpose of this test case is\\ to test the ability of the\\ program to reject a password\\ that has a space and fails\\
multiple rules\end{tabular}  \\
  \hline
  New Password & l H 1 ! \\
  \hline
  Current Password & 89hATs++ \\
  \hline
  Previous Password & FoolIng112=! \\
  \hline
  Expected Output & Rejected \\
  \hline
  Output & \\
  \hline
  Results & \multicolumn{1}{c|}{$\rule{0.5cm}{0.15mm}$Pass $\rule{0.5cm}{0.15mm}$Fail} \\ 
  \hline
  \end{tabular}
  \vspace{5mm}
  \begin{tabular}{|c|c|}
  \hline
  \multicolumn{2}{|c|}{Test Case 8} \\
  \hline
  Purpose & \begin{tabular}[c]{@{}l@{}}The purpose of this test case is\\ to test the ability of the\\ program to reject a password\\ that is similar to a previous one\end{tabular}  \\
  \hline
  New Password & 40!\#RacecaR \\
  \hline
  Current Password & \^{}002raceCAR, \\
  \hline
  Previous Password & WHat28;)pass \\
  \hline
  Expected Output & Rejected \\
  \hline
  Output & \\
  \hline
  Results & \multicolumn{1}{c|}{$\rule{0.5cm}{0.15mm}$Pass $\rule{0.5cm}{0.15mm}$Fail} \\ 
  \hline
  \end{tabular}
  \vspace{5mm}
  \begin{tabular}{|c|c|}
  \hline
  \multicolumn{2}{|c|}{Test Case 9} \\
  \hline
  Purpose & \begin{tabular}[c]{@{}l@{}}The purpose of this test case is\\ to test the ability of the\\ program to reject a password\\ that fails on multiple rules\end{tabular}  \\
  \hline
  New Password & 1aH ; \\
  \hline
  Current Password & \textgreater\textless33YOuqaaeqw2 \\
  \hline
  Previous Password & *GooD\^{}.,50 \\
  \hline
  Expected Output & Rejected \\
  \hline
  Output & \\
  \hline
  Results & \multicolumn{1}{c|}{$\rule{0.5cm}{0.15mm}$Pass $\rule{0.5cm}{0.15mm}$Fail} \\ 
  \hline
  \end{tabular}
  \vspace{5mm}
  \begin{tabular}{|c|c|}
  \hline
  \multicolumn{2}{|c|}{Test Case 10} \\
  \hline
  Purpose & \begin{tabular}[c]{@{}l@{}}The purpose of this test case is\\ to test the ability of the\\ program to accept a password\\ that is valid\end{tabular}  \\
  \hline
  New Password & 77WaitT!=gge \\
  \hline
  Current Password & \^{}002raceCAR, \\
  \hline
  Previous Password & FoolIng112=! \\
  \hline
  Expected Output & Accept \\
  \hline
  Output & \\
  \hline
  Results & \multicolumn{1}{c|}{$\rule{0.5cm}{0.15mm}$Pass $\rule{0.5cm}{0.15mm}$Fail} \\ 
  \hline
  \end{tabular}
  \vspace{5mm}
  \begin{tabular}{|c|c|}
  \hline
  \multicolumn{2}{|c|}{Test Case 11} \\
  \hline
  Purpose & \begin{tabular}[c]{@{}l@{}}The purpose of this test case is\\ to test the ability of the\\ program to reject a password\\ that is too short\end{tabular}  \\
  \hline
  New Password & 34ha\$=AT \\
  \hline
  Current Password & COrrect440\@\# \\
  \hline
  Previous Password & HousIng99!! \\
  \hline
  Expected Output & Rejected \\
  \hline
  Output & \\
  \hline
  Results & \multicolumn{1}{c|}{$\rule{0.5cm}{0.15mm}$Pass $\rule{0.5cm}{0.15mm}$Fail} \\ 
  \hline
  \end{tabular}
  \vspace{5mm}
  \begin{tabular}{|c|c|}
  \hline
  \multicolumn{2}{|c|}{Test Case 12} \\
  \hline
  Purpose & \begin{tabular}[c]{@{}l@{}}The purpose of this test case is\\ to test the ability of the\\ program to reject a password\\ that is similar to, backwards\end{tabular}  \\
  \hline
  New Password & ;:03Fires10001 \\
  \hline
  Current Password & Snake8,2Dl \\
  \hline
  Previous Password & 234+.;SErifant \\
  \hline
  Expected Output & Rejected \\
  \hline
  Output & \\
  \hline
  Results & \multicolumn{1}{c|}{$\rule{0.5cm}{0.15mm}$Pass $\rule{0.5cm}{0.15mm}$Fail} \\ 
  \hline
  \end{tabular}
  \vspace{5mm}
  \begin{tabular}{|c|c|}
  \hline
  \multicolumn{2}{|c|}{Test Case 13} \\
  \hline
  Purpose & \begin{tabular}[c]{@{}l@{}}The purpose of this test case is\\ to test the ability of the\\ program to reject a password\\ that has no numbers or special\\ characters\end{tabular}  \\
  \hline
  New Password & Housing\^{}.House \\
  \hline
  Current Password & orca28!\_Acrobatsstab \\
  \hline
  Previous Password & ValidPass\#!110 \\
  \hline
  Expected Output & Rejected \\
  \hline
  Output & \\
  \hline
  Results & \multicolumn{1}{c|}{$\rule{0.5cm}{0.15mm}$Pass $\rule{0.5cm}{0.15mm}$Fail} \\ 
  \hline
  \end{tabular}
  \vspace{5mm}
  \begin{tabular}{|c|c|}
  \hline
  \multicolumn{2}{|c|}{Test Case 14} \\
  \hline
  Purpose & \begin{tabular}[c]{@{}l@{}}The purpose of this test case is\\ to test the ability of the\\ program to reject a password\\ that has invalid characters\end{tabular}  \\
  \hline
  New Password & Test`erpa774*= \\
  \hline
  Current Password & lowerCASEpasswords!;123 \\
  \hline
  Previous Password & GUha891)) \\
  \hline
  Expected Output & Rejected \\
  \hline
  Output & \\
  \hline
  Results & \multicolumn{1}{c|}{$\rule{0.5cm}{0.15mm}$Pass $\rule{0.5cm}{0.15mm}$Fail} \\ 
  \hline
  \end{tabular}
  \vspace{5mm}
  \begin{tabular}{|c|c|}
  \hline
  \multicolumn{2}{|c|}{Test Case 15} \\
  \hline
  Purpose & \begin{tabular}[c]{@{}l@{}}The purpose of this test case is\\ to test the ability of the\\ program to reject a password\\ that has no letters and is similar to, forward\end{tabular}  \\
  \hline
  New Password &  \textgreater\textgreater!\@9100361=,\\
  \hline
  Current Password & ManySeats361=,\textgreater\$; \\
  \hline
  Previous Password & 234+.;SErifant \\
  \hline
  Expected Output & Rejected \\
  \hline
  Output & \\
  \hline
  Results & \multicolumn{1}{c|}{$\rule{0.5cm}{0.15mm}$Pass $\rule{0.5cm}{0.15mm}$Fail} \\ 
  \hline
  \end{tabular}
  \vspace{5mm}
  \begin{tabular}{|c|c|}
  \hline
  \multicolumn{2}{|c|}{Test Case 16} \\
  \hline
  Purpose & \begin{tabular}[c]{@{}l@{}}The purpose of this test case is\\ to test the ability of the\\ program to reject a password\\ that is too long and only has\\ upper case letters\end{tabular}  \\
  \hline
  New Password & THISSHOULDBEOKRIGHTPLEASEOKGOOD \\
  \hline
  Current Password & QuickPass42=! \\
  \hline
  Previous Password & 11\_MultiPass; \\
  \hline
  Expected Output & Rejected \\
  \hline
  Output & \\
  \hline
  Results & \multicolumn{1}{c|}{$\rule{0.5cm}{0.15mm}$Pass $\rule{0.5cm}{0.15mm}$Fail} \\ 
  \hline
  \end{tabular}
  \vspace{5mm}
  \begin{tabular}{|c|c|}
  \hline
  \multicolumn{2}{|c|}{Test Case 17} \\
  \hline
  Purpose & \begin{tabular}[c]{@{}l@{}}The purpose of this test case is\\ to test the ability of the\\ program to reject a password\\ that fails upper case and\\ numbers and is similar\\ to, backwards\end{tabular}  \\
  \hline
  New Password &  num\$!2!\#\@di\\
  \hline
  Current Password & 110VAlid\@\#!2!24 \\
  \hline
  Previous Password & QuickPass42=! \\
  \hline
  Expected Output & Rejected \\
  \hline
  Output & \\
  \hline
  Results & \multicolumn{1}{c|}{$\rule{0.5cm}{0.15mm}$Pass $\rule{0.5cm}{0.15mm}$Fail} \\ 
  \hline
  \end{tabular}
  \vspace{5mm}
  \begin{tabular}{|c|c|}
  \hline
  \multicolumn{2}{|c|}{Test Case 18} \\
  \hline
  Purpose & \begin{tabular}[c]{@{}l@{}}The purpose of this test case is\\ to test the ability of the\\ program to reject a password\\ that fails multiple, contains\\ a space and is similar\\ to, backwards\end{tabular}  \\
  \hline
  New Password & Civic1 ! \\
  \hline
  Current Password & vicLoaD1!9=ic \\
  \hline
  Previous Password & Snake8,2Dl \\
  \hline
  Expected Output & Rejected \\
  \hline
  Output & \\
  \hline
  Results & \multicolumn{1}{c|}{$\rule{0.5cm}{0.15mm}$Pass $\rule{0.5cm}{0.15mm}$Fail} \\ 
  \hline
  \end{tabular}
  \vspace{5mm}
  \begin{tabular}{|c|c|}
  \hline
  \multicolumn{2}{|c|}{Test Case 19} \\
  \hline
  Purpose & \begin{tabular}[c]{@{}l@{}}The purpose of this test case is\\ to test the ability of the\\ program to reject a password\\ that\end{tabular}  \\
  \hline
  New Password & RACECAR \\
  \hline
  Current Password & 36~RacecaR* \\
  \hline
  Previous Password & COrrect440\@\# \\
  \hline
  Expected Output & Rejected \\
  \hline
  Output & \\
  \hline
  Results & \multicolumn{1}{c|}{$\rule{0.5cm}{0.15mm}$Pass $\rule{0.5cm}{0.15mm}$Fail} \\ 
  \hline
  \end{tabular}
  \vspace{5mm}
  \begin{tabular}{|c|c|}
  \hline
  \multicolumn{2}{|c|}{Test Case 20} \\
  \hline
  Purpose & \begin{tabular}[c]{@{}l@{}}The purpose of this test case is\\ to test the ability of the\\ program to reject a password\\ that has an invalid character\end{tabular}  \\
  \hline
  New Password & ValidPass\#!110" \\
  \hline
  Current Password &  \\
  \hline
  Previous Password &  \\
  \hline
  Expected Output & Rejected \\
  \hline
  Output & \\
  \hline
  Results & \multicolumn{1}{c|}{$\rule{0.5cm}{0.15mm}$Pass $\rule{0.5cm}{0.15mm}$Fail} \\ 
  \hline
  \end{tabular}
  \vspace{5mm}
  \begin{tabular}{|c|c|}
  \hline
  \multicolumn{2}{|c|}{Test Case 21} \\
  \hline
  Purpose & \begin{tabular}[c]{@{}l@{}}The purpose of this test case is\\ to test the ability of the\\ program to reject a password\\ that fails multiple rules, contains a space\\ and is similar to, backwards\end{tabular}  \\
  \hline
  New Password & Kanakanak 1\_ \\
  \hline
  Current Password & Kana5507\^{}*kanak \\
  \hline
  Previous Password &  \\
  \hline
  Expected Output & Rejected \\
  \hline
  Output & \\
  \hline
  Results & \multicolumn{1}{c|}{$\rule{0.5cm}{0.15mm}$Pass $\rule{0.5cm}{0.15mm}$Fail} \\ 
  \hline
  \end{tabular}
  \vspace{5mm}
  \begin{tabular}{|c|c|}
  \hline
  \multicolumn{2}{|c|}{Test Case 22} \\
  \hline
  Purpose & \begin{tabular}[c]{@{}l@{}}The purpose of this test case is\\ to test the ability of the\\ program to reject a password\\ that contains a space and fails lower case\end{tabular}  \\
  \hline
  New Password & DAgo *\textgreater \\
  \hline
  Current Password &  \\
  \hline
  Previous Password &  \\
  \hline
  Expected Output & Rejected \\
  \hline
  Output & \\
  \hline
  Results & \multicolumn{1}{c|}{$\rule{0.5cm}{0.15mm}$Pass $\rule{0.5cm}{0.15mm}$Fail} \\ 
  \hline
  \end{tabular}
  \vspace{5mm}
  \begin{tabular}{|c|c|}
  \hline
  \multicolumn{2}{|c|}{Test Case 23} \\
  \hline
  Purpose & \begin{tabular}[c]{@{}l@{}}The purpose of this test case is\\ to test the ability of the\\ program to reject a password\\ that fails multiple rules, contains a space\\ and is similar to, forwards\end{tabular}  \\
  \hline
  New Password & aD765840!\@ =\textbar \\
  \hline
  Current Password &  \\
  \hline
  Previous Password & SHop5840!\@ \\
  \hline
  Expected Output & Rejected \\
  \hline
  Output & \\
  \hline
  Results & \multicolumn{1}{c|}{$\rule{0.5cm}{0.15mm}$Pass $\rule{0.5cm}{0.15mm}$Fail} \\ 
  \hline
  \end{tabular}
  \vspace{5mm}
  \begin{tabular}{|c|c|}
  \hline
  \multicolumn{2}{|c|}{Test Case 24} \\
  \hline
  Purpose & \begin{tabular}[c]{@{}l@{}}The purpose of this test case is\\ to test the ability of the\\ program to reject a password\\ that fails upper case and is too long\end{tabular}  \\
  \hline
  New Password & comp\_\$\@\&;;longer101017pass!~=*\textbackslash\textless \\
  \hline
  Current Password &  \\
  \hline
  Previous Password &  \\
  \hline
  Expected Output & Rejected \\
  \hline
  Output & \\
  \hline
  Results & \multicolumn{1}{c|}{$\rule{0.5cm}{0.15mm}$Pass $\rule{0.5cm}{0.15mm}$Fail} \\ 
  \hline
  \end{tabular}
  \vspace{5mm}
  \begin{tabular}{|c|c|}
  \hline
  \multicolumn{2}{|c|}{Test Case 25} \\
  \hline
  Purpose & \begin{tabular}[c]{@{}l@{}}The purpose of this test case is\\ to test the ability of the\\ program to reject a password\\ that fails special characters and\\ is similar to, forwards\end{tabular}  \\
  \hline
  New Password & LidOffadaffodil4 221 \\
  \hline
  Current Password &  \\
  \hline
  Previous Password & VA342*;daffodil \\
  \hline
  Expected Output & Rejected \\
  \hline
  Output & \\
  \hline
  Results & \multicolumn{1}{c|}{$\rule{0.5cm}{0.15mm}$Pass $\rule{0.5cm}{0.15mm}$Fail} \\ 
  \hline
  \end{tabular}
  \vspace{5mm}
  \begin{tabular}{|c|c|}
  \hline
  \multicolumn{2}{|c|}{Test Case 26} \\
  \hline
  Purpose & \begin{tabular}[c]{@{}l@{}}The purpose of this test case is\\ to test the ability of the\\ program to reject a password\\ that\end{tabular}  \\
  \hline
  New Password &  \\
  \hline
  Current Password &  \\
  \hline
  Previous Password &  \\
  \hline
  Expected Output & Rejected \\
  \hline
  Output & \\
  \hline
  Results & \multicolumn{1}{c|}{$\rule{0.5cm}{0.15mm}$Pass $\rule{0.5cm}{0.15mm}$Fail} \\ 
  \hline
  \end{tabular}
  \vspace{5mm}
  \begin{tabular}{|c|c|}
  \hline
  \multicolumn{2}{|c|}{Test Case 27} \\
  \hline
  Purpose & \begin{tabular}[c]{@{}l@{}}The purpose of this test case is\\ to test the ability of the\\ program to reject a password\\ that\end{tabular}  \\
  \hline
  New Password &  \\
  \hline
  Current Password &  \\
  \hline
  Previous Password &  \\
  \hline
  Expected Output & Rejected \\
  \hline
  Output & \\
  \hline
  Results & \multicolumn{1}{c|}{$\rule{0.5cm}{0.15mm}$Pass $\rule{0.5cm}{0.15mm}$Fail} \\ 
  \hline
  \end{tabular}
  \vspace{5mm}
  \begin{tabular}{|c|c|}
  \hline
  \multicolumn{2}{|c|}{Test Case 28} \\
  \hline
  Purpose & \begin{tabular}[c]{@{}l@{}}The purpose of this test case is\\ to test the ability of the\\ program to reject a password\\ that\end{tabular}  \\
  \hline
  New Password &  \\
  \hline
  Current Password &  \\
  \hline
  Previous Password &  \\
  \hline
  Expected Output & Rejected \\
  \hline
  Output & \\
  \hline
  Results & \multicolumn{1}{c|}{$\rule{0.5cm}{0.15mm}$Pass $\rule{0.5cm}{0.15mm}$Fail} \\ 
  \hline
  \end{tabular}
  \vspace{5mm}
  \begin{tabular}{|c|c|}
  \hline
  \multicolumn{2}{|c|}{Test Case 29} \\
  \hline
  Purpose & \begin{tabular}[c]{@{}l@{}}The purpose of this test case is\\ to test the ability of the\\ program to reject a password\\ that\end{tabular}  \\
  \hline
  New Password &  \\
  \hline
  Current Password &  \\
  \hline
  Previous Password &  \\
  \hline
  Expected Output & Rejected \\
  \hline
  Output & \\
  \hline
  Results & \multicolumn{1}{c|}{$\rule{0.5cm}{0.15mm}$Pass $\rule{0.5cm}{0.15mm}$Fail} \\ 
  \hline
  \end{tabular}
  \vspace{5mm}
  \begin{tabular}{|c|c|}
  \hline
  \multicolumn{2}{|c|}{Test Case 30} \\
  \hline
  Purpose & \begin{tabular}[c]{@{}l@{}}The purpose of this test case is\\ to test the ability of the\\ program to reject a password\\ that\end{tabular}  \\
  \hline
  New Password &  \\
  \hline
  Current Password &  \\
  \hline
  Previous Password &  \\
  \hline
  Expected Output & Rejected \\
  \hline
  Output & \\
  \hline
  Results & \multicolumn{1}{c|}{$\rule{0.5cm}{0.15mm}$Pass $\rule{0.5cm}{0.15mm}$Fail} \\ 
  \hline
  \end{tabular}
\end{center}


% Final section for the report
\section{Code}

\subsection{tester.c}

\lstinputlisting{code/tester.c}

\subsection{password\_checker.h}

\lstinputlisting{code/password_checker.h}

\subsection{password\_checker.c}

\lstinputlisting{code/password_checker.c}

\end{document}
